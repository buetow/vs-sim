\chapter{Ausblick}

Es wurde erfolgreich ein Simulator f�r die Simulation verteilter Systeme entwickelt. Der Simulator hat bereits 10 implementierte Protokolle zur Auswahl eingebaut. Zudem steht dem Gebraucher ein sehr komfortables Protokoll-API zur Verf�gung, womit der Entwicklung neuer Protokolle quasi keine Grenzen gesetzt sind.

Dar�ber hinaus verf�gt der Simulator �ber eine Vielzahl von sehr flexiblen Einstellungsm�glichkeiten. F�r jede Simulation lassen sich somit komplett andere Konfigurationen verwenden. Jeder beteiligte Prozess hat wiederum eigene lokale Einstellungen, wo sich auch jedes Protokoll f�r jeden Prozess separat einstellen l��t. Die Anzahl und Flexibilit�t der M�glichen Szenarien wird dadurch um einen sehr gro�en Faktor erweitert.

Mit dem Ereigniseditor gibt es eine komfortable M�glichkeit eigene Szenarien zu programmieren um sie anschlie�end zu Simulieren. Hierbei kann entweder auf die bereits enthaltenen Protokolle- oder auf selbst implementierte Protokolle zugegriffen werden. Alle Dazugeh�rigen Einstellungen und programmierten Ereignisse lassen sich vom Gebraucher f�r eine sp�tere Wiederverwendung plattformunabh�ngig abspeichern. Somit k�nnen auch abgespeicherte Szenarien beispielsweise an Kommilitonen weitergegeben werden oder f�r eine sp�tere Pr�sentierung zwischengespeichert werden. Mit dem Loggfilter lassen sich mithilfe von regul�ren Ausdr�cken nur die relevanten Loggnachrichten anzeigen, was die Analyse einer Simulation erheblich vereinfacht. Weitere Funktionalit�ten wie Lamport- und Vektor-Zeitstempel sowie Anti-Aliasing runden den Simulator ab. 

Durch den objektorientierten Aufbau ist der Simulator relativ einfach erweiterbar, was nicht nur das Protokoll-API betrifft. H�tte f�r diese Diplomarbeit noch mehr Zeit zur Verf�gung gestanden, dann k�nnten einige der folgenden Funktionen (hier in alphanumerisch sortierten Reihenfolge aufgelistet) auch eingebaut worden sein:

\begin{itemize}
	\setlength{\itemsep}{-2mm}
	\item Die Simulationsdauer beliebig lang machen k�nnen. Dazu m�sste \textit{VSSimulatorVisualisation} entlang der Zeitachse scrollbar gemacht werden, sodass der Benutzer f�r eine nachtr�gliche Betrachtung des Simulationsverlaufes zu jeder beliebigen Position zur�ckspringen kann.
	\item Eine Zoomfunktion f�r die Simulationsvisualisierung einbauen.
	\item Im Ereigniseditor selbst auch periodische Ereignisse programmierbar machen. Bisher kann nur jedes Ereignis separat programmiert werden oder auf Protokoll-Interne Wecker zur�ckgegriffen werden.
	\item Lamport- und Vektor-Zeitstempel als Ereigniseintrittskriterien verwenden k�nnen.
	\item Weitere Funktionalit�ten einbauen wie zum Beispiel das Anklicken einer Nachrichtenlinie, was zu einer Nachricht alle verf�gbaren Informationen anzeigt und diese gegebenenfalls vom Benutzer editiert werden k�nnen.
	\item Tiefere Schichten des OSI-Referenzmodells simulieren k�nnen, wie zum Beispiel TCP, UDP, IP, ...
\end{itemize}

Da der Simulator h�chstwahrscheinlich unter einer Open Source Lizenz freigegeben wird, und ich mich selbst sehr f�r die Entwicklung und Anwendung von Open Source Software interessiere, werden die einen oder anderen Funktionen nachtr�glich eingebaut werden. Kommilitonen werden auch herzlich dazu eingeladen sein sich an diesem Software-Projekt zu beteiligen. Als Vorbild sei hier der CPU-Simulator M32, der von Prof. O�mann an der Fachhochschule Aachen entwickelt wurde, genannt. Hier existieren bereits einige Erweiterungen und Verbesserungen der Ursprungsversion, die von den Studenten angefertigt wurden. F�r die Entwicklung/Erweiterung wurde keine propriet�re Software verwendet, sodass jeder kostenlosen Zugriff auf die dazugeh�rigen Tools h�tte.
