\chapter{Ausblick}

Es wurde erfolgreich ein Simulator f\"{u}r die Simulation verteilter Systeme entwickelt. Der Simulatur hat bereits 10 implementierte Protokolle zur Auswahl eingebaut. Zudem steht dem Gebraucher ein sehr komfortables Protokoll-API zur Verf\"{u}gung, womit der Entwicklung neuer Protokolle quasi keine Grenzen gesetzt sind.

Dar\"{u}berhinaus verf\"{u}gt der Simulator \"{u}ber eine Vielzahl von sehr flexiblen Einstellungsm\"{o}glichkeiten. F\"{u}r jede Simulation lassen sich somit komplett andere Konfigurationen verwenden. Jeder beteiligte Prozess hat wiederum eingene lokale Einstellungen, wo sich auch jedes Protokoll f\"{u}r jeden Prozess separat einstellen l\"{a}�t. Die Anzahl und Flexibilit\"{a}t der M\"{o}glichen Szenarien wird dadurch um einen sehr gro�en Faktor vergr\"{o}�ert.

Mit dem Ereigniseditor gibt es eine komfortable M\"{o}glichkeit eigene Szenarien zu programmieren und zu Simulieren. Hierbei kann entweder auf die bereits enthaltenen Protokolle- oder auf selbst implementierte Protokolle zugegriffen werden. Alle Dazugeh\"{o}rigen Einstellungen und programmierten Ereignisse lassen sich vom Gebraucher f\"{u}r eine sp\"{a}tere Wiederverwendung platformunabh\"{a}ngig abspeichern. Somit k\"{o}nnen auch abgespeicherte Szenarien beispielsweise an Komilitonen weitergegeben werden oder f\"{u}r eine sp\"{a}tere Pr\"{a}sentierung zwischengespeichert werden. Mit dem Loggfilter lassen sich mithilfe von regul\"{a}ren Ausdr\"{u}cken nur die relevanten Loggnachrichten anzeigen, was die Analyse einer Simulation erheblich vereinfacht. Weitere Funktionalit\"{a}ten wie Lamport- und Vektor-Zeitstempel sowie Anti-Aliasing ruden den Simulator ab. 

Durch den objektorientierten Aufbau ist der Simulator relativ einfach erweiterbar, was nicht nur f\"{u}r das Protokoll-API betrifft. H\"{a}tte f\"{u}r diese Diplomarbeit noch mehr Zeit zur Verf\"{u}gung gestanden, dann k\"{o}nnten einige der folgenden Funktionen (hier in alphanumerisch sortierten Reihenfolge aufgelistet) auch eingebaut worden sein:

\begin{itemize}
	\setlength{\itemsep}{-2mm}
	\item Die Simulationsdauer beliebig lang machen k\"{o}nnen. Dazu m\"{u}sste \textit{VSSimulatorVisualisation} entlang der Zeitachse scrollbar gemacht werden, sodass der Benutzer f\"{u}r eine nachtr\"{a}gliche Betrachtung des Simulationsverlaufes zu jeder beliebigen Position zur\"{u}ckspringen kann.
	\item Eine Zoomfunktion f\"{u}r die Simulationsvisualisierung.
	\item Im Ereigniseditor selbst auch periodische Ereignisse programmierbar machen. Bisher kann nur jedes Ereignis separat programmiert werden oder auf Protokoll-Interne Wecker zur\"{u}ckgegriffen werden.
	\item Lamport- und Vektor-Zeitstempel f\"{u}r Ereigniseintrittskriterien verwenden.
	\item Weitere Funktionalit\"{a}ten wie zum Beispiel das Anklicken einer Nachrichtenlinie, was zu einer Nachicht alle verf\"{u}gbaren Informationen anzeigt und diese gegebenenfalls vom Benutzer editiert werden k\"{o}nnen.
\end{itemize}

Da der Simulator h\"{o}chstwahrscheinlich unter einer Open Source Lizenz freigegeben wird, und ich mich selbst sehr f\"{u}r die Entwicklung und Anwendung von Open Source Software interessiere, werden die einen oder anderen Funktionen nachtr\"{a}glich eingebaut werden. Komilitonen werden auch herzlich dazu eingeladen werden sich an diesem Software-Projekt zu beteiligen. Als Vorbild sei hier der CPU-Simulator M32, der von Prof. Ossmann an der Fachhochschule Aachen entwickelt wurde, genannt. Hier existieren bereits einige Erweiterungen Verbesserungen, die von den Studenten angefertigt wurden. F\"{u}r die Entwicklung/Erweiterung wurde keine properit\"{a}re Software verwendet, sodass jeder kostenlosen Zugriff auf die dazugeh\"{o}rigen Tools h\"{a}tte.
