\chapter{Ausblick}

Es wurde erfolgreich ein Simulator f�r die Simulation verteilter Systeme entwickelt. Der Simulator hat bereits 10 implementierte Protokolle zur Auswahl eingebaut. Zudem steht dem Anwender ein sehr komfortables Protokoll-API zur Verf�gung, womit der Entwicklung neuer Protokolle quasi keine Grenzen gesetzt sind.

Dar�ber hinaus verf�gt der Simulator �ber eine Vielzahl von sehr flexiblen Einstellungsm�glichkeiten. F�r jede Simulation lassen sich somit komplett andere Konfigurationen verwenden. Jeder beteiligte Prozess hat wiederum eigene lokale Einstellungen, wo sich auch jedes Protokoll f�r jeden Prozess separat einstellen l�sst. Die Anzahl und Flexibilit�t der m�glichen Szenarien wird dadurch um einen sehr gro�en Faktor erweitert.

Mit dem Ereigniseditor gibt es eine komfortable M�glichkeit eigene Szenarien zu programmieren um sie anschlie�end zu Simulieren. Hierbei kann entweder auf die bereits enthaltenen Protokolle oder auf selbst implementierte Protokolle zugegriffen werden. Alle dazugeh�rigen Einstellungen und programmierten Ereignisse lassen sich vom Anwender f�r eine sp�tere Wiederverwendung plattformunabh�ngig abspeichern. Somit k�nnen auch abgespeicherte Szenarien beispielsweise an Kommilitonen weitergegeben werden oder f�r eine sp�tere Pr�sentierung zwischengespeichert werden. Durch den Logfilter lassen sich mit Hilfe von regul�ren Ausdr�cken nur die relevanten Lognachrichten anzeigen, was die Analyse einer Simulation erheblich vereinfacht. Weitere Funktionen wie Lamport- und Vektor-Zeitstempel sowie Anti-Aliasing runden den Simulator ab. 

Durch den objektorientierten Aufbau ist der Simulator relativ einfach erweiterbar, was nicht nur das Protokoll-API betrifft. Insgesamt wurde an den meisten Stellen darauf geachtet, dass zu einem sp\"{a}teren Zeitpunkt Erweiterungen einflie�en k\"{o}nnten. Insbesondere soll die Serialisierung von Objekten r\"{u}ckw\"{a}rtskompatibel bleiben, da sonst bei jeder neuen Simulatorversion alle Simulationen erneut angelegt und abgespeichert werden m\"{u}ssten. 

H�tte f�r diese Diplomarbeit noch mehr Zeit zur Verf�gung gestanden, dann h\"{a}tten einige der folgenden Funktionen (hier in alphanumerisch sortierter Reihenfolge aufgelistet) auch Einzug halten k\"{o}nnen:

\begin{itemize}
	\item Die M\"{o}glichkeit Protokolle zu entwickeln ohne den kompletten Quelltext des Simulators vorliegen zu haben. Protokollklassen als separate Bibliothek einzubinden, die dynamisch geladen werden k\"{o}nnen.
	\item Die Simulationsdauer beliebig lang machen zu k�nnen. Dazu m�sste die Klasse \textit{VSSimulatorVisualisation} entlang der Zeitachse scrollbar gemacht werden, so dass der Benutzer f�r eine nachtr�gliche Betrachtung des Simulationsverlaufes zu jeder beliebigen Position zur�ckspringen kann.
	\item Eine Zoomfunktion f�r die Simulationsvisualisierung einzubauen.
	\item Im Ereigniseditor selbst auch periodische Ereignisse programmierbar zu machen. Bisher kann nur jeder Ereigniseintritt separat programmiert werden oder auf Protokoll-Interne Wecker zur�ckgegriffen werden.
	\item Lamport- und Vektor-Zeitstempel als Ereigniseintrittskriterien verwenden zu k�nnen.
	\item Tiefere Schichten des OSI-Referenzmodells simulieren k�nnen, wie zum Beispiel TCP, UDP, IP, ...
	\item Weitere Funktionen einzubauen, wie zum Beispiel das Anklicken einer Nachrichtenlinie, was zu der jeweiligen Nachricht alle verf�gbaren Informationen anzeigt und welche gegebenenfalls vom Benutzer editiert werden k�nnen.
\end{itemize}

Da der Simulator h�chstwahrscheinlich unter einer Open Source Lizenz freigegeben wird, werden die einen oder anderen Funktionen nachtr�glich eingebaut werden. Kommilitonen werden auch herzlich dazu eingeladen sein sich an diesem Software-Projekt zu beteiligen. Als Vorbild sei hier der CPU-Simulator M32 \cite{M32}, der von Prof. O�mann an der Fachhochschule Aachen entwickelt wurde, genannt. Hier existieren bereits einige Erweiterungen und Verbesserungen der Ursprungsversion, die von den Studenten angefertigt wurden. F�r die Entwicklung des VS-Simulators wurde keine propriet�re Software verwendet, so dass jeder kostenlosen Zugriff auf die dazugeh�rigen Tools hat.
