% %	Diploma thesis template 2005
%
%       author: lukas.silberbauer(at)gmx.at
%       based upon  "Diplomarbeit mit LaTeX" by Tobias Erbsland
%
%       published under the terms of
%
%  ----------------------------------------------------------------------------
%  "THE BEER-WARE LICENSE":
%  <lukas.silberbauer(at)gmx.at> wrote this file. As long as you retain this 
%	notice you can do whatever you want with this stuff. If we meet some day, 
%	and you think this stuff is worth it, you can buy me a beer in return. 
%  ----------------------------------------------------------------------------

%
% header.tex
%
\documentclass[%
	pdftex,%              PDFTex verwenden
	a4paper,%             A4 Papier
	oneside,%             Einseitig
	bibtotocnumbered,%    Literaturverzeichnis nummeriert einf�gen
	idxtotoc,%            Index ins Verzeichnis einf�gen
	halfparskip,%         Europ�ischer Satz mit abstand zwischen Abs�tzen
	chapterprefix,%       Kapitel anschreiben als Kapitel
	%headsepline,%         Linie nach Kopfzeile
	%footsepline,%         Linie vor Fusszeile
	12pt%                 Gr��ere Schrift, besser lesbar am bildschrim
]{scrbook}


%
% Paket f�r die Indexerstellung.
%
\usepackage{makeidx}

%
% Paket f�r �bersetzungen ins Deutsche
%
\usepackage[german, english]{babel}

%
% Pakete um Latin1 Zeichnens�tze verwenden zu k�nnen und die dazu
%  passenden Schriften.
%
\usepackage[latin1]{inputenc}
\usepackage[T1]{fontenc}

%
% Paket zum Erweitern der Tabelleneigenschaften
%
\usepackage{array}

%
% Paket um Grafiken einbetten zu k�nnen
%
\usepackage{graphicx}

%
% Spezielle Schrift verwenden.
%
\renewcommand{\encodingdefault}{T1}
%\renewcommand{\familydefault}{goudysans}
\renewcommand{\familydefault}{\sfdefault}


%
% Zeilenabstand einstellen
%
\usepackage{setspace}
\onehalfspacing
%\doublespacing


%\setlength{\baselineskip}{24pt}
%\renewcommand{\baselinestretch}{1.5} 



%
% define variables
%
\def\maintitle#1{\gdef\maintitle{#1}}
\def\subtitle#1{\gdef\subtitle{#1}}

%
% Zeilenumbruch bei Bildbeschreibungen.
%
\setcapindent{1em}

%
% kopf und fusszeilen
%
\pagestyle{headings}

%
% mathematische symbole aus dem AMS Paket.
%
\usepackage{amsmath}
\usepackage{amssymb}

%
% Type 1 Fonts f�r bessere darstellung in PDF verwenden.
%
\usepackage{mathptmx}           % Times + passende Mathefonts
\usepackage[scaled=.92]{helvet} % skalierte Helvetica als \sfdefault
\usepackage{courier}            % Courier als \ttdefault

%
% Paket um Textteile drehen zu k�nnen
%
\usepackage{rotating}


%
% F�r Acronyme
%
\usepackage{acronym}

%
% Package f�r Farben im PDF
%
\usepackage{color}

%
% Paket f�r Links innerhalb des PDF Dokuments
%
\definecolor{LinkColor}{rgb}{0,0,0.5}

\usepackage[%
pdfauthor={Paul B\"{u}tow},
bookmarks=true, % PDF bookmarks allowed. NB! The level depth of bookmarks is the same as in the TOC.
unicode=true, % PDF bookmarks in Unicode.
bookmarksnumbered=true, % Section numbers in PDF bookmarks.
bookmarksopenlevel=1, % The open level in PDF bookmarks.
hyperindex=true, % Hyperlinked index.
colorlinks=true, % Links are marked as coloured text, not coloured box.
linkcolor=linkc, % The colour for in-document links (e.g. in the table of contents).
citecolor = citec, % The colour for bibliographic citations.
urlcolor=urlc, % The colour for hyperlinks to the Net.
pdfpagelayout=OneColumn % Continuous page scrolling.
]{hyperref}
\hypersetup{colorlinks=true,%
	linkcolor=LinkColor,%
	citecolor=LinkColor,%
	filecolor=LinkColor,%
	menucolor=LinkColor,%
	pagecolor=LinkColor,%
	urlcolor=LinkColor}

%
% Paket um Listings sauber zu formatieren.
%
\usepackage[savemem]{listings}
\lstloadlanguages{TeX}

%
% ---------------------------------------------------------------------------
% Listing Definationen f�r PHP Code
%
\definecolor{lbcolor}{rgb}{0.85,0.85,0.85}
\lstset{language=[LaTeX]TeX,
	numbers=left,
	stepnumber=1,
	numbersep=5pt,
	numberstyle=\tiny,
	breaklines=true,
	breakautoindent=true,
	postbreak=\space,
	tabsize=2,
	basicstyle=\ttfamily\footnotesize,
	showspaces=false,
	showstringspaces=false,
	extendedchars=true,
	backgroundcolor=\color{lbcolor}}

% ---------------------------------------------------------------------------
% Neue Umgebungen
% ---------------------------------------------------------------------------

\newenvironment{ListChanges}%
	{\begin{list}{$\diamondsuit$}{}}%
	{\end{list}}

%
% Index erzeucgen
%
\makeindex

%
% EOF
%
